\documentclass[10pt,aspectratio=169]{beamer}

% All the boilerplate is in deslides.sty
\usepackage{deslides}

\author{Ji\v{r}\'i Lebl}

\institute[OSU]{%
Oklahoma State University%
%Departemento pri Matematiko de Oklahoma {\^S}tata Universitato%
}

\title{23. The Laplace transform, part 1\\(Notes on Diffy Qs, 6.1)}

\date{}

\begin{document}

\begin{frame}
\titlepage

%\bigskip

\begin{center}
The textbook: \url{https://www.jirka.org/diffyqs/}
\end{center}
\end{frame}

\begin{frame}
Solving differential equations is hard!
It would be a lot easier if there were no derivatives.

\medskip
\pause

Laplace transform to the rescue:
\[
\begin{tikzcd}[ampersand replacement=\&]
\boxed{\text{algebraic equation}} \arrow[r, "\text{easy}"] \&
\boxed{\text{algebraic solution}} \arrow[d, "\text{inverse Laplace transform}"] \\
\boxed{\text{differential equation}}
\arrow[r, "\text{hard}"] \arrow[u, "\text{Laplace transform}"] \&
\boxed{\text{solution to differential equation}}
\end{tikzcd}
\]

\pause

Other applications: electrical circuit analysis, NMR
spectroscopy, signal processing, etc.

\medskip
\pause

Laplace transform is similar to the Fourier transform and Fourier series.

\medskip
\pause

What it does: Laplace takes a function of time $t$ and gives a function of
``frequency'' $s$:
\[
\sL \bigl\{ f(t) \bigr\} = F(s)
\]
\pause
Notation: We will use capital letters to represent the Laplace transforms.

\end{frame}

\begin{frame}
In particular, consider
\[
m x''(t) + c x'(t) + k x(t) = f(t) .
\]
Think of $t$ as time, $f(t)$ as the input signal, and $x(t)$ as the output.

\medskip
\pause

We will take $t=0$ to be the initial time and consider $x(t)$
only for $t \geq 0$.

\medskip
\pause

\textbf{Example:} In the mass-spring system, the external force is the input and
the position of the mass is the output.

\medskip
\pause

\textbf{Example:}
Or in the RLC circuit, the change in voltage on the electric source
was the input and the current in the circuit is the output.

\medskip
\pause

\textbf{Solution procedure:}

We transform the input $f(t)$ into $F(s)$.

\medskip
\pause

We transform the equation to get an equation in the transforms $X(s)$ and $F(s)$
that we solve by algebra.

\medskip
\pause

Then we invert the transform of $X(s)$ to find the output $x(t)$.

\end{frame}

\begin{frame}
\textbf{Definition:}  Suppose that $f(t)$ is a function of time $t \geq 0$.
Then
\[
\mathcal{L} \bigl\{ f(t) \bigr\} =
F(s) \overset{\text{def}}{=} \int_0^\infty e^{-st} f(t) \, dt .
\]
\pause

\textbf{Example:}
Consider $f(t) = 1$
\begin{equation*}
\mathcal{L} \{1\} = \int_0^\infty e^{-st} \, dt
\pause
=
\left[ \frac{e^{-st}}{-s} \right]_{t=0}^\infty
\pause
=
\lim_{h\to\infty}
\left[ \frac{e^{-st}}{-s} \right]_{t=0}^h
\pause
=
\lim_{h\to\infty}
\left( \frac{e^{-sh}}{-s} - \frac{1}{-s} \right)
\pause
= \frac{1}{s} .
\end{equation*}
Limit exists for $s > 0$
\pause
\wthus
$\mathcal{L} \{1\}$ only defined for $s > 0$.

\medskip
\pause

\textbf{Example:}
Consider
$f(t) = e^{-at}$
\begin{equation*}
\mathcal{L} \bigl\{e^{-at}\bigr\}
= \int_0^\infty e^{-st} e^{-at} \, dt
\pause
= \int_0^\infty e^{-(s+a)t} \, dt
\pause
=
\left[ \frac{e^{-(s+a)t}}{-(s+a)} \right]_{t=0}^\infty
\pause
= \frac{1}{s+a} .
\end{equation*}
Limit exists for $s+a > 0$  ~(that is, $s > -a$)
\pause
\wthus
$\mathcal{L} \bigl\{e^{-at}\bigr\}$ only defined for $s+a > 0$.

\end{frame}

\begin{frame}
\textbf{Example:}
Consider $f(t) = t$, and assume $s > 0$
\begin{equation*}
\mathcal{L} \{t\}
= \int_0^\infty e^{-st} t \, dt
\pause
=
\left[ \frac{-te^{-st}}{s} \right]_{t=0}^\infty
+
\frac{1}{s}
\int_0^\infty e^{-st} \,dt
\pause
=
0
+
\frac{1}{s}
\left[ \frac{e^{-st}}{-s} \right]_{t=0}^\infty
\pause
=
\frac{1}{s^2} .
\end{equation*}
%Limit only exists if $s > 0$.

\medskip
\pause

\textbf{Example:}  Consider the
\emph{unit step function}, sometimes the
\emph{Heaviside function}
\begin{equation*}
u(t) = 
\begin{cases}
0 & \text{if } \; t < 0 , \\
1 & \text{if } \; t \geq 0 ,
\end{cases}
\qquad
\text{or more generally}
\qquad
\pause
u(t-a) = 
\begin{cases}
0 & \text{if } \; t < a , \\
1 & \text{if } \; t \geq a .
\end{cases}
\end{equation*}
\pause
The formulation with a number $a \geq 0$ is more useful:
$u(t-a)$ is $0$ for $t < a$ and $1$ for $t \geq a$.
\pause
Again assume $s > 0$.
\pause
\begin{equation*}
\mathcal{L} \bigl\{ u(t-a) \bigr\}
=
\int_0^{\infty} e^{-st} u(t-a) \, dt
\pause
=
\int_a^{\infty} e^{-st} \, dt
\pause
=
\left[ \frac{e^{-st}}{-s} \right]_{t=a}^\infty
\pause
=
\frac{e^{-as}}{s} .
\end{equation*}
%\pause
%Limit only exists if $s > 0$ (and $a \geq 0$).

\end{frame}

\begin{frame}

Here are some useful Laplace transforms ($C$, $\omega$, and $a$ are constants):

\begin{center}
\begin{tabular}{@{}cclcc@{}}
\toprule
$f(t)$ & $\mathcal{L} \bigl\{ f(t) \bigr\} = F(s)$ & $\qquad\quad$ &
$f(t)$ & $\mathcal{L} \bigl\{ f(t) \bigr\} = F(s)$ \\
\midrule
$C$ & $\dfrac{C}{s}$
& &
$\sin (\omega t)$ & $\dfrac{\omega}{s^2+\omega^2}$
\\[12pt]
$t$ & $\dfrac{1}{s^2}$
& &
$\cos (\omega t)$ & $\dfrac{s}{s^2+\omega^2}$
\\[12pt]
$t^2$ & $\dfrac{2}{s^3}$
& &
$\sinh (\omega t)$ & $\dfrac{\omega}{s^2-\omega^2}$
\\[12pt]
$t^3$ & $\dfrac{6}{s^4}$
& &
$\cosh (\omega t)$ & $\dfrac{s}{s^2-\omega^2}$
\\[12pt]
$t^n$ & $\dfrac{n!}{s^{n+1}}$
& &
$u(t-a)$ & $\dfrac{e^{-as}}{s}$
\\[12pt]
$e^{-at}$ & $\dfrac{1}{s+a}$
& & &
\\[12pt]
\bottomrule
\end{tabular}
\end{center}

\pause

\textbf{Exercise:}
Verify the table.

\end{frame}

\begin{frame}
The transform is defined by an integral and integral is linear, so
if $C$ is a constant
\begin{equation*}
\mathcal{L} \bigl\{ C f(t) \bigr\}
\pause
=
\int_0^\infty e^{-st} C f(t) \,dt
\pause
=
C \int_0^\infty e^{-st} f(t) \,dt
\pause
=
C \mathcal{L} \bigl\{ f(t) \bigr\} .
\end{equation*}
\pause
Similarly, if $A$ and $B$ are constants:
\begin{equation*}
\mathcal{L} \bigl\{ A f(t) + B g(t) \bigr\} =
A \mathcal{L} \bigl\{ f(t) \bigr\} +
B \mathcal{L} \bigl\{ g(t) \bigr\} .
\end{equation*}

%\pause
%\textbf{Exercise:} Verify this statement.

\medskip
\pause

\textbf{Example:}
\[
\mathcal{L} \bigl\{
2 + 9t + \cos(2t) + 5 e^{-2t}
\bigr\}
\pause
=
\mathcal{L} \bigl\{
2
\bigr\}
+ 9
\mathcal{L} \bigl\{
t
\bigr\}
+
\mathcal{L} \bigl\{
\cos(2t)
\bigr\}
+ 5
\mathcal{L} \bigl\{
e^{-2t}
\bigr\}
\pause
=
%\frac{2}{s}
%+ 9
%\frac{1}{s^2}
%+
%\frac{2}{s^2+4}
%+ 5
%\frac{1}{s+2}
%=
\frac{2}{s}
+
\frac{9}{s^2}
+
\frac{2}{s^2+4}
+
\frac{5}{s+2} .
\]
\pause

\textbf{Caution:}
In general 
\begin{equation*}
\mathcal{L} \bigl\{ f(t) g(t) \bigr\} \not=
\mathcal{L} \bigl\{ f(t) \bigr\}
\mathcal{L} \bigl\{ g(t) \bigr\} .
\end{equation*}

\medskip
\pause

\textbf{Remark:}
Not all functions have a Laplace transform. \quad E.g.,
~
$\frac{1}{t}$, ~$\tan t$, or ~ $e^{t^2}$

\end{frame}


\end{document}
