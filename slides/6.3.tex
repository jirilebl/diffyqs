\documentclass[10pt,aspectratio=169]{beamer}

% All the boilerplate is in deslides.sty
\usepackage{deslides}

\author{Ji\v{r}\'i Lebl}

\institute[OSU]{%
Oklahoma State University%
%Departemento pri Matematiko de Oklahoma {\^S}tata Universitato%
}

\title{26. Convolution (Laplace transform)\\(Notes on Diffy Qs, 6.3)}

\date{}

\begin{document}

\begin{frame}
\titlepage

%\bigskip

\begin{center}
The textbook: \url{https://www.jirka.org/diffyqs/}
\end{center}
\end{frame}

\begin{frame}
Laplace transform of a product is not the product of the transforms.

\medskip
\pause

Suppose $f(t)$ and $g(t)$ are defined for $t \geq 0$.  Define
the \emph{convolution} as a new function $f * g$:
\[
(f * g)(t) \overset{\text{def}}{=}
\int_0^t f(\tau) g(t-\tau) \, d\tau .
\]

\pause

\emph{Remark:}  It is the same formula as
$\int_{-\infty}^\infty f(\tau) g(t-\tau) \, d\tau$
assuming $f(t)=g(t)=0$ for $t < 0$.

\medskip
\pause

\textbf{Example:}
Let $f(t) = e^t$ and $g(t) = t$ for $t \geq 0$.  Then 
\[
(f*g)(t)
=
\int_0^t e^\tau (t-\tau) \, d\tau
\pause
=
\text{(integration by parts)}
=
e^t - t - 1 .
\]

\medskip
\pause

\textbf{Example:}
Take $f(t) = \sin (\omega t)$ and $g(t) = \cos (\omega t)$ for $t \geq 0$.

Then (use the identity
$\cos (\theta) \sin (\psi) =
\frac{1}{2} \, \bigl( \sin (\theta + \psi) - \sin (\theta - \psi) \bigr)$)
\[
(f*g)(t)
=
\int_0^t  \sin ( \omega \tau ) \,
\cos \bigl( \omega (t-\tau) \bigr) \, d\tau
\pause
=
\int_0^t
\frac{1}{2} \, \bigl( \sin (\omega t) - \sin (\omega t - 2 \omega \tau
) \bigr) \, d\tau
\]
Apply the identity
\begin{equation*}
\cos (\theta) \sin (\psi) =
\frac{1}{2} \, \bigl( \sin (\theta + \psi) - \sin (\theta - \psi) \bigr) ,
\end{equation*}
to get
\begin{equation*}
\begin{split}
& =
\left[ \frac{1}{2} \, \tau  \sin (\omega t) + \frac{1}{4\omega} \, \cos (2 \omega \tau -
\omega t) \right]_{\tau=0}^t
\\
& = \frac{1}{2} \, t \sin (\omega t) .
\end{split}
\end{equation*}
The formula holds only for $t \geq 0$.  The functions $f$, $g$,
and $f*g$ are undefined for $t < 0$.

\end{frame}

\begin{frame}

Convolution has many properties that make it behave like a product.
Let $c$ be a constant and $f$, $g$, and $h$ be functions.  Then it is a
calculus exercise to show
\begin{align*}
& f * g = g * f , \\
& (c f) * g = f * (c g) = c (f*g) , \\
& (f+g) * h = f * h + g * h , \\
& ( f * g ) * h = f * ( g * h ) .
\end{align*}
The most interesting property for us is the following theorem.

\begin{theorem}
If $f(t)$ and $g(t)$ are of exponential order, then
so is $(f*g)(t)$ and
\begin{equation*}
\mathcal{L} \bigl\{ (f*g)(t) \bigr\}
=
\mathcal{L} \left\{ \int_0^t f(\tau) g(t-\tau) \, d\tau \right\}
=
\mathcal{L} \bigl\{ f(t) \bigr\} \mathcal{L} \bigl\{ g(t) \bigr\} .
\end{equation*}
\end{theorem}

In other words, the Laplace transform of a convolution is the product
of the Laplace transforms.  The simplest way to use this result is in
reverse.

\begin{example}
Suppose we have the function of $s$
defined by
\begin{equation*}
\frac{1}{(s+1)s^2} = 
\frac{1}{s+1}\,
\frac{1}{s^2} .
\end{equation*}
We recognize the two entries of \tableref{ltd:table}.  That is,
\begin{equation*}
\mathcal{L}^{-1} 
\left\{
\frac{1}{s+1} \right\}
= e^{-t}
\qquad \text{and} \qquad
\mathcal{L}^{-1} 
\left\{
\frac{1}{s^2} \right\} 
= t.
\end{equation*}
Therefore,
\begin{equation*}
\mathcal{L}^{-1}
\left\{
\frac{1}{s+1}\,
\frac{1}{s^2} \right\}
=
\int_0^t
\tau e^{-(t-\tau)} \,d\tau
=
e^{-t}+t-1 .
\end{equation*}
The calculation of the integral involved an integration by parts.
\end{example}

\end{frame}

\begin{frame}

{Solving ODEs}

The next example demonstrates the full power of the convolution and
the Laplace transform.  We can give the solution to
the forced oscillation problem for any forcing function as a definite
integral.

\begin{example} \label{example:undampedforcedbylaplacearbitrhs}
Find the solution to
\begin{equation*}
x'' + \omega_0^2 x = f(t) , \quad x(0) = 0, \quad x'(0) = 0 ,
\end{equation*}
for an arbitrary function $f(t)$.

We first apply the Laplace transform to the equation.  Denote
the transform of $x(t)$ by $X(s)$ and the transform of $f(t)$ by
$F(s)$ as usual.  We get
\begin{equation*}
s^2 X(s) + \omega_0^2 X(s) = F(s) ,
\end{equation*}
or in other words,
\begin{equation*}
X(s) = \frac{1}{s^2+ \omega_0^2} F(s).
\end{equation*}
Recall that $H(s) = \frac{1}{s^2+ \omega_0^2}$ is the transfer function.
We know
\begin{equation*}
{\mathcal{L}}^{-1} \left\{
\frac{1}{s^2+ \omega_0^2}
\right\} = 
\frac{\sin (\omega_0 t)}{\omega_0} .
\end{equation*}
Therefore,
\begin{equation*}
x(t) = 
\int_0^t
\frac{\sin (\omega_0 \tau)}{\omega_0}
f(t-\tau) \, d\tau ,
\end{equation*}
or, if we reverse the order,
\begin{equation*}
x(t) = 
\int_0^t
f(\tau) 
\frac{\sin \bigl( \omega_0 (t-\tau) \bigr)}{\omega_0} \, d\tau .
\end{equation*}
\end{example}

Notice one more feature of the example above.
We can now see how Laplace transform
handles resonance.  Suppose that $f(t) =
\cos (\omega_0 t)$.  Then
\begin{equation*}
x(t) = 
\int_0^t
\frac{\sin (\omega_0 \tau)}{\omega_0} \,
\cos \bigl( \omega_0 (t-\tau) \bigr) \, d\tau
=
\frac{1}{\omega_0}
\int_0^t
\sin ( \omega_0 \tau ) \,
\cos \bigl(\omega_0 (t-\tau) \bigr) \, d\tau .
\end{equation*}
We have computed the convolution of sine and cosine in
\exampleref{ltc:convsincosex}.  Hence
\begin{equation*}
x(t) =
\left(
\frac{1}{\omega_0}
\right) \,
\left(
\frac{1}{2} \,
t
\sin ( \omega_0 t )
\right)
=
\frac{1}{2 \omega_0} \,
t
\sin ( \omega_0 t ).
\end{equation*}
Note the $t$ in front of the sine.  The solution, therefore, grows without
bound as $t$ gets large, meaning we get resonance.

The general idea here is that if $H(s)$ is the transfer function, then
$X(s)=H(s)F(s)$.
If we find the $h(t) = {\mathcal{L}}^{-1}\bigl\{ H(s) \bigr\}$, then
\begin{equation*}
x(t)
= {\mathcal{L}}^{-1}\bigl\{ X(s) \bigr\}
= {\mathcal{L}}^{-1}\bigl\{ F(s)H(s) \bigr\}
= (f * h)(t)
= \int_0^t f(\tau) h(t-\tau) \, d\tau .
\end{equation*}
Hence,
we can solve any constant coefficient equation with an arbitrary forcing
function $f(t)$ as a definite integral using convolution.
A definite integral, rather than a closed form solution, is usually enough
for most practical purposes.  It is
not hard to numerically evaluate a definite integral.

\end{frame}

\begin{frame}

{Volterra integral equation}

A common integral equation\index{integral equation}
is the \emph{Volterra integral equation}%
\footnote{Named for the Italian mathematician
\href{https://en.wikipedia.org/wiki/Vito_Volterra}{Vito Volterra}
(1860--1940).}
\begin{equation*}
x(t) = f(t) + \int_0^t g(t-\tau) x(\tau) \, d\tau ,
\end{equation*}
where $f(t)$ and $g(t)$ are known functions and $x(t)$ is an unknown we
wish to solve for.
To find $x(t)$,
we apply the Laplace transform to the equation to obtain 
\begin{equation*}
X(s) = F(s) + G(s) X(s) ,
\end{equation*}
where $X(s)$, $F(s)$, and $G(s)$ are the Laplace transforms of $x(t)$, $f(t)$, and
$g(t)$ respectively.  We find
\begin{equation*}
X(s) = \frac{F(s)}{1-G(s)} .
\end{equation*}
To find $x(t)$, we now need to find the 
inverse Laplace transform of $X(s)$.

\begin{example}
Solve
\begin{equation*}
x(t) =  e^{-t} + \int_0^t \sinh(t-\tau) x(\tau) \, d\tau .
\end{equation*}

We apply Laplace transform to obtain
\begin{equation*}
X(s) = \frac{1}{s+1} + \frac{1}{s^2-1} X(s) ,
\end{equation*}
or
\begin{equation*}
X(s) = \frac{\frac{1}{s+1}}{1- \frac{1}{s^2-1}}
=
\frac{s-1}{s^2 - 2}
=
\frac{s}{s^2 - 2}
-
\frac{1}{s^2 - 2} .
\end{equation*}
It is not hard to apply \tablevref{lt:table} to find
\begin{equation*}
x(t) = \cosh \bigl( \sqrt{2} \, t \bigr) -
\frac{1}{\sqrt{2}} \sinh \bigl( \sqrt{2}\, t \bigr).
\end{equation*}
\end{example}

\end{frame}


\end{document}
