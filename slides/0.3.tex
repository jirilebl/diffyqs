\documentclass[10pt,aspectratio=169]{beamer}

% All the boilerplate is in deslides.sty
\usepackage{deslides}

\author{Ji\v{r}\'i Lebl}

\institute[OSU]{%
Oklahoma State University%
%Departemento pri Matematiko de Oklahoma {\^S}tata Universitato%
}

\title{3. Classification of differential equations\\(Notes on Diffy Qs, 0.3)}

\date{}

\begin{document}

\begin{frame}
\titlepage

%\bigskip

\begin{center}
The textbook: \url{https://www.jirka.org/diffyqs/}
\end{center}
\end{frame}

\begin{frame}
There are many types of differential equations.

\medskip
\pause

Most broadly:

\begin{itemize}
\item
\pause
\emph{Ordinary differential equations} (ODE):

Only one independent variable.
\item
\pause
\emph{Partial differential equations} (PDE):

Several independent variables, using partial derivatives.
\end{itemize}

\end{frame}

\begin{frame}

\textbf{Examples:}

\medskip

ODE:

\medskip
\pause

\quad\makebox[0pt][l]{$\displaystyle
\frac{d y}{dt} = ky$}\hspace*{2in}%
(Exponential growth)

\medskip
\pause

\quad\makebox[0pt][l]{$\displaystyle
\frac{d y}{dt} = k(A-y)$}\hspace*{2in}%
(Newton's law of cooling)

\medskip
\pause

\quad\makebox[0pt][l]{$\displaystyle
m \frac{d^2 x}{dt^2} + c \frac{dx}{dt} + kx = f(t)$}\hspace*{2in}%
(Mechanical vibrations)

\medskip
\pause

PDE:

\medskip
\pause

\quad\makebox[0pt][l]{$\displaystyle
\frac{\partial y}{\partial t} + c \frac{\partial y}{\partial x} = 0$}\hspace*{2in}%
(Transport equation)

\medskip
\pause

\quad\makebox[0pt][l]{$\displaystyle
\frac{\partial u}{\partial t} = \frac{\partial^2 u}{\partial x^2}$}\hspace*{2in}%
(Heat equation)
\medskip
\pause

\quad\makebox[0pt][l]{$\displaystyle
\frac{\partial^2 u}{\partial t^2} = \frac{\partial^2 u}{\partial x^2} +
\frac{\partial^2 u}{\partial y^2}$}\hspace*{2in}%
(Wave equation in 2 dimensions)

\end{frame}

\begin{frame}

If there is more than one equation it is a
\emph{system of differential equations}.

\medskip
\pause

\textbf{Examples:}

\medskip
\pause

A \emph{system of ordinary differential equations} (system of ODE):
\[
y' = x , \qquad x' = y .
\]

\medskip
\pause

Maxwell's equations for electromagnetics are a
\emph{system of partial differential equations}

(system of PDE):
\begin{align*}
& \nabla \cdot \vec{D} = \rho, & & \nabla \cdot \vec{B} = 0 , \\
& \nabla \times \vec{E} = - \frac{\partial \vec{B}}{\partial t}, &
& \nabla \times \vec{H} = \vec{J} + \frac{\partial \vec{D}}{\partial t} .
\end{align*}
(Note: 
divergence $\nabla \cdot$ and 
curl $\nabla \times$ are written in partial derivatives in $x,y,z$.)

\end{frame}

\begin{frame}
The highest order derivative that appears is the \emph{order} of the
equation (or system).

\medskip
\pause

\quad\makebox[0pt][l]{$\displaystyle
\frac{d y}{dt} = ky
$}\hspace*{2in}\quad is a first order ODE equation.

\medskip
\pause

\quad\makebox[0pt][l]{$\displaystyle
\frac{d^2 y}{dx^2} + \frac{dy}{dx} + y = \sin(x)
$}\hspace*{2in}\quad is a second order ODE equation.

\medskip
\pause

\quad\makebox[0pt][l]{$\displaystyle
a^4 \frac{\partial^4 y}{\partial x^4} + \frac{\partial^2 y}{\partial t^2} = 0
$}\hspace*{2in}\quad is a fourth order PDE equation.

\bigskip
\pause

\textbf{Remark:} The most common equations in physics are first and second order.

\end{frame}

\begin{frame}
An equation is \emph{linear} if the dependent variable(s) and their
derivatives appear linearly.

\pause

(``linearly'': only first powers, not multiplied together, divided, or composed
with functions such as $\sin$ or $\exp$.)

\medskip
\pause

An ODE of order $n$ can be put into the form:
\[
a_n(x) \frac{d^n y}{dx^n} + 
a_{n-1}(x) \frac{d^{n-1} y}{dx^{n-1}} + 
\cdots
+
a_{1}(x) \frac{dy}{dx}
+
a_{0}(x) y = b(x) .
\]

\medskip
\pause

E.g., linear 2${}^\text{nd}$ order ODE:
\vspace*{-5pt}
\[
e^x \frac{d^2 y}{dx^2} + 
\sin(x) \frac{d y}{dx} + 
x^2 y
=
\frac{1}{x}
\]
\pause
Note: The dependence on $x$ need not be linear, only the dependence on
$y$.

\medskip
\pause

E.g., \textbf{non}linear 2$^{\text{nd}}$ order PDE (Burger's equation):
\[
\frac{\partial y}{\partial t} + 
y \frac{\partial y}{\partial x} =
\nu \frac{\partial^2 y}{\partial x^2} .
\]

\medskip
\pause

E.g., \textbf{non}linear 1$^{\text{st}}$ order ODE:
\vspace*{-5pt}
\[
\frac{dx}{dt} = x^2
\]

\medskip
\pause

\textbf{Remark:} Nonlinear equations are notoriously difficult to handle.

\end{frame}

\begin{frame}
A linear equation is \emph{homogeneous} if all terms depend on the dependent variable.

\pause
Otherwise, it is \emph{nonhomogeneous} or \emph{inhomogeneous}.

\medskip
\pause

\textbf{Examples:}

\pause
\medskip

Homogeneous:

\pause
\vspace*{-12pt}
\hspace*{1.5in}$\displaystyle
e^x \frac{d^2 y}{dx^2} + 
\sin(x) \frac{d y}{dx} + 
x^2 y
=
0
$

\pause
\medskip

\hspace*{1.5in}$\displaystyle
a^4 \frac{\partial^4 y}{\partial x^4} + \frac{\partial^2 y}{\partial t^2} = 0
$

\pause
\medskip

\hspace*{1.5in}$\displaystyle
a_n(x) \frac{d^n y}{dx^n} + 
a_{n-1}(x) \frac{d^{n-1} y}{dx^{n-1}} + 
\cdots
+
a_{1}(x) \frac{dy}{dx}
+
a_{0}(x) y = 0
$

\pause
\medskip

Inhomogeneous:

\pause
\vspace*{-12pt}
\hspace*{1.5in}$\displaystyle
e^x \frac{d^2 y}{dx^2} + 
\sin(x) \frac{d y}{dx} + 
x^2 y
=
\frac{1}{x}
$

\pause
\medskip

\hspace*{1.5in}$\displaystyle
\frac{dx}{dt} + x + t = 0 
$

\pause
\medskip

The inhomogeneity is often some ``outside input'' into the physical
system.

\medskip
\pause

We solve an inhomogeneous equation using the solution to the corresponding
homogeneous equation.
\end{frame}

\begin{frame}
A linear equation has
\emph{constant coefficients} if the coefficients are constants (except for
perhaps the inhomogeneity).
\pause
A general constant coefficient linear ODE:
\[
a_n \frac{d^n y}{dx^n} + 
a_{n-1} \frac{d^{n-1} y}{dx^{n-1}} + 
\cdots
+
a_{1} \frac{dy}{dx}
+
a_{0} y = b(x) ,
\]
$a_0, a_1, \ldots, a_n$ are constants, $b$ may depend on $x$.

\medskip
\pause

Finally, an equation (or system) is \emph{autonomous} if the equation does
not depend on the independent variable at all.

\medskip
\pause

E.g., ~$\dfrac{dx}{dt}=x^2$~
is autonomous, but ~$\dfrac{dx}{dt}=xt$~ is not.

\medskip
\pause

Autonomous equations often appear when the setup is
independent of time.
\end{frame}

\begin{frame}
\textbf{Examples:} (see if you can guess the properties before they are
revealed)

\medskip
\pause

$\bullet$\quad$\displaystyle
\frac{d y}{dt} = ky$

\medskip
\pause

First order, autonomous, linear, homogeneous, constant coefficient ODE
equation.

\medskip
\pause

$\bullet$\quad$\displaystyle
\frac{d y}{dt} = k(A-y)$,
\qquad ($A\not=0$)

\medskip
\pause

First order, autonomous, linear, inhomogeneous, constant coefficient ODE
equation.

\medskip
\pause

$\bullet$\quad$\displaystyle
\frac{d^2 x}{dt^2} + \frac{dx}{dt} + x = \sin(t)$

\medskip
\pause

Second order, linear, inhomogeneous, constant coefficient ODE equation.

\medskip
\pause

$\bullet$\quad$\displaystyle
\theta'' + \sin(\theta) = 0$

\medskip
\pause

Second order, autonomous, nonlinear ODE equation.

\medskip
\pause

$\bullet$\quad$\displaystyle
\frac{\partial u}{\partial t} + \frac{\partial v}{\partial x} = x^2$
\qquad
$\displaystyle
\frac{\partial u}{\partial t} - \frac{\partial v}{\partial x} = xt$

\medskip
\pause

First order, linear, inhomogeneous, constant coefficient PDE system.

\end{frame}

\end{document}
